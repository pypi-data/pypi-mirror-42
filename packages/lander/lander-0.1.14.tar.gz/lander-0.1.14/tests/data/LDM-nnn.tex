\documentclass[DM,lsstdraft,toc]{lsstdoc}

% Package imports go here

% Local commands go here

\def \theDocTitle {Title of document}

\title[Short title]{\theDocTitle}

\author{%
A.~Author,
B.~Author,
and
C.~Author}

\setDocRef{LDM-nnn}
\date{\today}

% Optional
\setDocCurator{The Curator of this Document}


\setDocAbstract{%
This document demonstrates how to use the LSST \LaTeX\ class files to make Data Management
documents. Build this document in the normal way, making sure that the class file is
available in the \LaTeX\ load path.
}

% Change history defined here. Will be inserted into
% correct place with \maketitle
% OLDEST FIRST: VERSION, DATE, DESCRIPTION, OWNER NAME
\setDocChangeRecord{%
\addtohist{1}{2017-09-10}{Initial release. Based on Gaia examples.}{Tim Jenness}
\addtohist{2}{yyyy-mm-dd}{Future changes}{Future person}
}

\begin{document}

% Create the title page
% Table of contents will be added automatically if "toc" class option
% is used.
\maketitle

\section{Introduction}

Now write your document as you would normally write it.
Different citation schemes are supported, and the default bibliography style is declared by the class.

\verb|\citellp|: \citellp{LPM-17, LSE-30} \\
\verb|\citell|: (SRD; \citell{LPM-17,LSE-29}) \\
\verb|\citep[][]|: \citep[e.g.,][are interesting]{LPM-17,LSE-29} \\
\verb|\cite|: \cite{LPM-17,LSE-29}

Font checking: \texttt{Fixed width font}.

Math checking: $A = \pi r^2 \mathrm{(math roman)}$

% Include all the relevant bib files.
% lsst is for DocuShare and DMTN entries.
% refs_ads is for entries coming from ADS.
\bibliography{lsst,refs,books,refs_ads}

\end{document}
