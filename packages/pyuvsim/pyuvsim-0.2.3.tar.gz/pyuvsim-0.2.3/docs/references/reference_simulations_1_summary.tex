\documentclass{article}

\usepackage{hyperref}
\usepackage{tabularx}
\usepackage{graphicx}
\usepackage{subcaption}
\usepackage{float}
\usepackage{geometry}
\geometry{landscape, margin=1in}
\date{August 8, 2018}


\title{pyuvsim Reference Simulations \\ version 1}

 
\begin{document}
\maketitle
The simulations here are intended as reference point for with existing simulators. Pyuvsim phase, amplitude, spectral performance, spatial geometry etc are covered by multiple tests which helps build confidence in the accuracy of these reference simulation.

These tests span axes chosen to highlight important physical directions (time, frequency, baselines, sources). Some tests are done with different beams of interest. The total number of data points is constrained by the current level of optimization and availability of computing resources. \texttt{pyuvsim} is currently running on the Oscar cluster at Brown University. The next steps will be to cleanup, restructure, and optimize our code to allow for more sources/times/frequencies/baselines to be simulated at a given time. More documentation on pyuvsim can be found at \url{https://pyuvsim.readthedocs.io}.

\section*{Overview}

% Please add the following required packages to your document preamble:
% \usepackage{graphicx}
\begin{table}[h!]
\hspace*{-.75in}
\resizebox{.95\paperwidth}{!}{%
\begin{tabular}{|r|c|c|c|c|c|l|}
\hline
\textbf{Obsparam File} & obsparam\_ref\_1.1\_uniform.yaml & obsparam\_ref\_1.2\_gauss.yaml & obsparam\_ref\_1.2\_uniform.yaml & obsparam\_ref\_1.3\_gauss.yaml & obsparam\_ref\_1.3\_uniform.yaml & obsparam\_ref\_1.4.yaml \\ \hline
\textbf{Catalog} & mock\_catalog\_heratext\_2458098.38824015.txt & two\_distant\_points\_2458098.38824015.txt & two\_distant\_points\_2458098.38824015.txt & letter\_R\_12pt\_2458098.38824015.txt & letter\_R\_12pt\_2458098.38824015.txt & gleam.vot \\ \hline
\textbf{Ntimes} & 1 & 86400 & 86400 & 2 & 2 & 1 \\ \hline
\textbf{Nfreqs} & 1 & 1 & 1 & 64400 & 64400 & 1 \\ \hline
\textbf{Layout} & MWA\_nocore & Baseline-lite & Baseline-lite & Baseline-lite & Baseline-lite & 5km triangle \\ \hline
\textbf{Beam} & uniform & 11$^\circ$ FWHM gaussian & uniform & 11$^\circ$ FWHM gaussian & uniform & 11$^\circ$ FWHM gaussian \\ \hline
\textbf{Results File} & ref\_1.1\_uniform.uvfits & ref\_1.2\_gauss.uvfits & ref\_1.2\_uniform.uvfits & ref\_1.3\_gauss.uvfits & ref\_1.3\_uniform.uvfits & ref\_1.4\_uniform.uvfits \\ \hline
\end{tabular}%
}
\end{table}

Each column above summarizes the key parameters of our first round of reference simulations. 

For a full description of how antenna layouts, instrument configuration, and catalogs are all written into parameter files, please see the documentation at \url{https://pyuvsim.readthedocs.io/en/latest/parameter_files.html}. As a quick summary: antenna layout is specified by a csv file, overall array configuration and primary beam assignments are defined by a yaml file, and catalogs are defined either by VOTable files or csv files. The "obsparam" files encapsulate all other configuration parameters, including the catalog, telescope configuration, array layout, and time/frequency array structures, as well as output filing information and any additional UVData parameters that may be desired.

The sections below will describe each configuration in more detail.

 
\section*{Catalogs}

\begin{itemize}
\item[] mock\_catalog\_heratext\_2458098.38824015.txt:
     This is a set of point sources near zenith at JD 2458098.38824015 for an observer at the HERA location. They spell out the word "HERA" from east to west across the sky with the tops of the letters to the north.

\item[] two\_distant\_points\_2458098.38824015.txt:
     This is two points near the opposite horizons at the specified julian date for an observer at the HERA location. This is chosen so that one source will rise and cross the sky for a given time, and we can see if the other sets appropriately.

\item[] letter\_R\_12pt\_2458098.38824015.txt:
     This is just the letter "R" from the HERA text catalog. It was chosen to have a smaller catalog with a recognizable orientation on the sky.

\item[] gleam.vot:
     The GLEAM catalog. It's too large to fit on github, so it's not included in the data directory, but it's on lustre.

\end{itemize}


\section*{Layouts}

\begin{figure*}[h]
\centering
\begin{subfigure}[b]{0.3\textwidth}
\includegraphics[width=\textwidth]{mwa88_layout.png}
\caption{MWA\_nocore layout}
\end{subfigure}
\begin{subfigure}[b]{0.3\textwidth}
\includegraphics[width=\textwidth]{bllite.png}
\caption{Baseline-lite}
\end{subfigure}
\begin{subfigure}[b]{0.3\textwidth}
\includegraphics[width=\textwidth]{fivekm_triangle.png}
\caption{5km triangle}
\end{subfigure}
\end{figure*}


\begin{enumerate}
\item[] \textbf{MWA\_nocore}:

     This is the MWA128 layout with the core 40 antennas removed (88 antennas remain). The layout is written in mwa\_nocore\_layout.csv and the telescope configuration is in mwa88\_nocore\_config.yaml. The configuration specifies that all antennas have the unphysical "uniform" beam.

\item[] \textbf{baseline\_lite}:

     This consists of a right triangle of antennas with an additional antenna sqrt(2) meters off of the center of the hypotenuse. This provides a perfectly N-S and E-W and diagonal baselines, as well as some that don't perfectly fit the symmetry.

     The layout is in baseline\_lite.csv, and the bl\_lite\_gauss.yaml and bl\_lite\_uniform.yaml files respectively assign gaussian and uniform beams to all four antennas.

\item[] \textbf{5km triangle}:

     An isosceles triangle consisting of two 5km baselines. Layout and configuration (gaussian beam) are in 5km\_triangle\_layout.csv and 5km\_triangle\_config.yaml.
\end{enumerate}


All three layouts are shown in the images below:


\section*{Beams}

 

Only two types of primary beams were used in these simulations: The uniform beam, which has unit response at all alt/az, and an 11° fwhm Gaussian beam. Both are AnalyticBeam objects,

 
\section*{Location}

All simulations chose the HERA site as their telescope\_location, for simplicity. This is lat/lon/alt (-30.72153°, 21.42831°, 1073.0 m). This includes the MWA128-like array.

 
\section*{Paths on lustre}

\begin{enumerate}
\item[] pyuvsim files: /lustre/aoc/projects/hera/ref\_sim/
\item[] PRISim files:  /lustre/aoc/projects/hera/djacobs/prisim\_ref/
\end{enumerate}



\end{document}
